\documentclass[12pt]{article}
\usepackage[utf8]{inputenc}
\usepackage{amsmath}


\begin{document}
\title{Elementos de programación Python 1}
\author{Reyna Cornelio}
\date{25 Enero 2016}
\maketitle



 
 A continuación se presentaran algunos problemas resueltos  Con la finalidad de familiarizarse con el entorno de Python. Se adjuntan los codigos que se utilizaron para poder resolver los problemas
 
\subsection*{Caída de un cuerpo} 

Se deja caer una pelota desde el techo de una torre de altura h. Se desea saber la altura de la pelota respecto a la torre a un determinado tiempo después de haber sido dejada caer.
Para este primer problema se copió el código que se presenta a continuación

\begin{verbatim}
# -*- coding: utf-8 -*-
"""
Created on Mon Jan 25 20:06:38 2016

@author: Reyna
"""
h = float(input("Proporciona la altura de la torre: "))
t = float(input("Ingrese el tiempo: "))
s = 0.5*9.81*t**2
print("La altura de la pelota es", h-s, "metros")

\end{verbatim}

Este programa fue sencillo de realizar el problema surgió al modificarlo para que nos diera el tiempo, para ello, se realizó el siguiente código.

\begin{verbatim}
inserte código aquí
\end{verbatim}




\subsubsection*{Coordenadas polares}
Este programa calcula los valores a coordenadas polares. Con este código se producirá uno similiar pra poder calcular las coordenadas esféricas.
\begin{verbatim}
from math import sin,cos,pi

r = float(input("introduce r: " ))

d = float(input("Ingresa theta en grados: "))

theta = d*pi/180

x = r*cos(theta)

y = r*sin(theta)

print("x=",x,"y=",y)
\end{verbatim}

\subsubsection*{Coordenadas esféricas}
\begin{verbatim}
r = float(input("Introduce r: " ))

d = float(input("Introduce theta en grados: "))

d2 = float(input("Introduce phi en grados: " ))

theta = d*pi/180 
phi= d*pi/180 

x = r*sin(theta)*cos(phi)

y = r*sin(theta)*cos(phi)

z = r*cos(theta)

print ("x =",x,"y =",y, "z =",z)
\end{verbatim}

\subsubsection*{Números pares e impares}

Este programa me resultó algo confuso, debido a que, no leí bien las instrucciones, sin embargo, una vez entendida las indicaciones fue sencillo de realizar
\begin{verbatim}
n = int(input("Enter an integer: " ))

if n%2==0:
    
    print("even")
    
else:
    
    print("odd")

print ("Enter two integers, on even, one odd.")

m = int(input("Enter the first integer: 1") )

n = int(input("Enter the second integer: "))
    
while (m+n)%2==0:
    
    print ("One must be even the other odd.")
    
    m = int(input("Enter the first integer: " ))
    n = int(input("Enter the second integer: "))
    
print("The numbers you chose are",m,"and",n)
\end{verbatim}

\subsubsection*{Fibonnacci}

En la terminal no se muestran muchos número de la secuencia, se debe de suponer al rango que se pone.

\begin{verbatim}
f1,f2 = 1,1

while f2<1000:
    print(f2)
    f1,f2 = f2,f1+f2
\end{verbatim}
 \end{document}